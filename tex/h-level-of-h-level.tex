\documentclass[11pt]{amsart}

%% Input handling:
\usepackage[utf8]{inputenc}  % to allow unicode in source

%% AMS and other general math packages:
\usepackage{amsthm}
\usepackage{amsfonts}
\usepackage{amsmath}
\usepackage{amssymb}
\usepackage{mathtools} % for \mathclap, used to center extra-wide diagrams.

%% For Coq code
\usepackage{listings}
\def\lstlanguagefiles{defManSSR.tex}
\lstset{language=SSR}


%% General style packages:
\usepackage{xcolor}
\definecolor{darkgreen}{rgb}{0,0.45,0}
\definecolor{darkred}{rgb}{0.75,0,0}
\definecolor{darkblue}{rgb}{0,0,0.6}

\definecolor{dkviolet}{rgb}{0.6,0,0.8}
\definecolor{dkblue}{rgb}{0,0.1,0.5}
\definecolor{lightblue}{rgb}{0,0.5,0.5}
\definecolor{dkgreen}{rgb}{0,0.4,0}
\definecolor{dk2green}{rgb}{0.4,0,0}


\usepackage[colorlinks,citecolor=darkgreen,linkcolor=darkred,urlcolor=darkblue]{hyperref}
\usepackage{breakurl}
%\usepackage{mathpazo}
\usepackage{enumerate} % for customising enumerated lists

%% For syntax of type theory:
% \usepackage{mathpartir}

%% Graphics and diagrams packages:
\usepackage{tikz}
\usetikzlibrary{arrows}
\usepackage[all]{xy}
\xyoption{2cell}
\xyoption{curve}
\UseTwocells
% \input{diagxy}  % for better inline arrows
% \input{xy-in-tikz} % for arrow tips in TikZ matching xypic’s style
% \tikzset{>=xyto}
\pgfarrowsdeclarealias{c}{c}{right hook}{left hook}
\pgfarrowsdeclarealias{c'}{c'}{left hook}{right hook}

\newbox\pbbox
\setbox\pbbox=\hbox{\xy \POS(65,0)\ar@{-} (0,0) \ar@{-} (65,65)\endxy}
\def\pb{\save[]+<3.5mm,-3.5mm>*{\copy\pbbox} \restore}
% A pullback marker for xymatrix diagrams.
% Usage: add “\pb” in the matrix node of the pullback object, as for instance:
% \xymatrix{ A \ar[r] \ar[d] \pb & B \ar[d] \\ C \ar[r] & D}
%
% To adjust placement in individual diagrams, instead of “\pb” include the
% code following “\def\pb”, and play around with the positioning “<3mm,-3mm>”.

%% Theorem environment declarations (using amsthm):
\theoremstyle{plain}
\newtheorem{theorem}{Theorem}
\newtheorem{axiom}[theorem]{Axiom}
\newtheorem*{theoremstar}{Theorem}
\newtheorem{fact}[theorem]{Fact}
\newtheorem{proposition}[theorem]{Proposition}
\newtheorem{lemma}[theorem]{Lemma}
\newtheorem{corollary}[theorem]{Corollary}

\theoremstyle{definition}
\newtheorem{definition}[theorem]{Definition}
\newtheorem{convention}[theorem]{Convention}
\newtheorem{example}[theorem]{Example}
\newtheorem{examples}[theorem]{Examples}
\newtheorem{notation}[theorem]{Notation}
\newtheorem{remark}[theorem]{Remark}
\newtheorem{idea}[theorem]{Idea}

% Type-theoretic:
\newcommand{\Eq}{\mathsf{Eq}}
\newcommand{\HLevel}{\mathsf{HLevel}}
\newcommand{\hProp}{\mathsf{hProp}}
\newcommand{\Id}{\mathsf{Id}}
\newcommand{\isofhlevel}{\mathsf{isofhlevel}}
\newcommand{\Nat}{\mathsf{Nat}}
\newcommand{\U}{\mathsf{U}}

% Other:
\newcommand{\oftype}{\! : \!}
\newcommand*{\into}{\ensuremath{\lhook\joinrel\relbar\joinrel\rightarrow}}

\makeatletter
\renewcommand{\paragraph}{\@startsection{paragraph}{4}{0mm}{-0.5\baselineskip}{-1ex}{\bf}}
\makeatother


\begin{document}
\title{H-Level $n$ is of h-level $n\mbox{+}1$}

\author{Benedikt Ahrens}
%\address[Benedikt Ahrens]{Institute for Advanced Study, Princeton}
%\email{ahrens@ias}

\author{Chris Kapulkin}
%\address[Chris Kapulkin]{Institute for Advanced Study, Princeton; and University of Pittsburgh}
%\email{krk56@ias.edu}

\date{December 14th, 2012}
\maketitle

In (traditional) mathematics it is well-known that $n$-groupoids and their isomorphisms form an $(n\mbox{+}1)$-groupoid. Here, we prove an analogous theorem in Voevodsky's Univalent Foundations, namely, that the type of types of h-level $n$ is itself of h-level $n+1$.\footnote{We do not claim much originality: a proof of this result can be found in \cite[Section~8.7]{warren-pelayo:univalent-foundations-paper}.}

We start by fixing a type-theoretic universe $\U$ closed under standard constructors (of importance to us are: $\prod$, $\sum$, $=$). Define the type
 \[\HLevel(n) := \sum_{X : \U} \ \isofhlevel(n)(X). \]

Our main theorem is:

\begin{theorem}\label{main_thm}
 The type $\HLevel(n)$ is of h-level $n+1$.
\end{theorem}

Before proving the theorem, we state a helpful lemma.

\begin{lemma}\label{Id_of_Sigma}
 Let $P \colon \U \to \hProp$ and let $(X, p), (X', p') \oftype \sum\limits_{X : \U} P(X)$. Then:
 \[ \big( (X, p) =_{\sum\limits_{X : \U} P(X)} (X', p')\big) \simeq (X =_{\U} X').\]
\end{lemma}

\begin{proof}
 We have:
 \begin{equation*}\begin{split}
 ((X, p) =_{\sum\limits_{X : \U} P(X)} (X', p')) & \simeq \sum_{z : X = X'} (z_!p =_{P(X')} p') \\
  & \simeq (X =_{\U} X')
 \end{split}
 \end{equation*}
 where the first equivalence follows from the fact that a path in a $\sum$-type corresponds to a pair of paths: one in the base type, and one in the fiber; and the second is an immediate consequence of $P(X')$ being an h-Prop.
\end{proof}

\begin{remark}
 Note that from Lemma \ref{Id_of_Sigma} one can deduce what may be called {\em Univalence for propositions on the universe} i.e. for any $P \colon \U \to \hProp$ there is a weak equivalence:
 \[ \Big( (X, p) =_{\sum\limits_{X : \U} P(X)} (X', p')\Big) \simeq \Eq (X, X'). \]
\end{remark}

\begin{proof}[Proof of Theorem \ref{main_thm}]
 Let $(X, p), (X', p') \oftype \HLevel(n)$. We need to show that the type:
 \[(X, p) = (X', p') \]
 is of h-level $n$. Since for any natural number $n$ and any type $X$ the type: $\isofhlevel(n)(X)$ is an h-Prop, we may apply Lemma \ref{Id_of_Sigma} with $P := \isofhlevel(n)$. By the Univalence Axiom, to show that $(X = X')$ is of h-level $n$, it suffices to show that that $\Eq(X, X')$ is of h-level $n$. We have an inclusion (a map of h-level $1$):
 \[\Eq (X, X') \into (X \rightarrow X').\]
 Thus it suffices to show that $X \rightarrow X'$ is of h-level $n$. Since h-levels are preserved under $\prod$ and hence under the arrow type, this reduces to an assumption that $X'$ is of h-level $n$.
\end{proof}

\paragraph*{About the formalization in Univalent Foundations}
We have proved this result formally in the proof assistant \textsf{Coq}, based on V.\ Voevodsky's 
library \textsf{Foundations} (TODO : insert reference with URL).
In the following we present the formal equivalents to the above statements. 
We write \lstinline!a == b! for $a = b$, and \lstinline!weq X Y! denotes the type of weak 
equivalences from $X$ to $Y$. The total space associated to a fibration $B$ is denoted by 
\lstinline!total2 (fun x => B x)!.

Voevodsky's library \textsf{Foundations} provides many results we needed in order to prove the theorem. However, one important
result was missing: given a fibration, the space of paths in its total space is equivalent to 
  the space of pairs of paths in the base space and paths in the fiber:
\begin{lstlisting}
Theorem total_paths_equiv (B : UU -> hProp)(x y : total2 (fun x => B x)):
  weq (x == y) 
      (total2 (fun p : pr1 x == pr1 y => transportf _ p (pr2 x) == pr2 y )).
\end{lstlisting}
We imported this result from the library \textsf{HoTT} (TODO : insert reference with URL); see the file
{\verb+auxiliary_lemmas_HoTT.v+} of our repository (TODO : insert reference with URL).

In the file {\verb+HLevel_n_is_of_hlevel_Sn.v+} we prove the above statements.
Lemma \ref{Id_of_Sigma} reads as
\begin{lstlisting}
Lemma Id_p_weq_Id (P : UU -> hProp)(X X' : UU)(pX : P X)(pX' : P X'): 
  weq ((tpair _ X pX) == (tpair (fun x => P x) X' pX')) (X == X').
\end{lstlisting}
In the proof of the main theorem \ref{main_thm} we apply the preceding lemma in order to 
reduce the goal to the following statement:
\begin{lstlisting}
Lemma isofhlevelpathspace : forall n : nat, forall X Y : UU,
    isofhlevel n X -> isofhlevel n Y -> isofhlevel n (X == Y).
\end{lstlisting}
This statement is proved by distinguishing $n = 0$ and $n > 0$; in the library this is done in 
two separate lemmas, called \lstinline!isofhlevel0pathspace! and \lstinline!isofhlevelSnpathspace!,
respectively.

Altogether, the two files comprise about 80 lines of specification and 150 lines of proof, 40 percent of which
are used for the translation of \lstinline!total_paths_equiv! from the \textsf{HoTT} library to 
\textsf{Foundations}.
The proof (along with the necessary files from Voevodsky's Foundations repository) can be found at:
\begin{center}
 \texttt{https://github.com/benediktahrens/HLevel\_n\_is\_of\_hlevel\_Sn}
\end{center}

\bibliographystyle{abbrvnat}
\bibliography{references}

\end{document} 