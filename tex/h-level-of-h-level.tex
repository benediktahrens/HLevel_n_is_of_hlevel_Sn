\documentclass[11pt]{scrartcl}

%% Input handling:
\usepackage[utf8]{inputenc}  % to allow unicode in source

%%  KOMA SCRIPT font settings
\setkomafont{title}{\rmfamily\bfseries}
\setkomafont{section}{\rmfamily\bfseries\Large}

%% AMS and other general math packages:
\usepackage{amsthm}
\usepackage{amsfonts}
\usepackage{amsmath}
\usepackage{amssymb}
\usepackage{mathtools} % for \mathclap, used to center extra-wide diagrams.

%% For Coq code
\usepackage{listings}
\def\lstlanguagefiles{defManSSR.tex}
\lstset{language=SSR}

%% For links
\usepackage{url}

%% General style packages:
\usepackage{xcolor}
\definecolor{darkgreen}{rgb}{0,0.45,0}
\definecolor{darkred}{rgb}{0.75,0,0}
\definecolor{darkblue}{rgb}{0,0,0.6}

\definecolor{dkviolet}{rgb}{0.6,0,0.8}
\definecolor{dkblue}{rgb}{0,0.1,0.5}
\definecolor{lightblue}{rgb}{0,0.5,0.5}
\definecolor{dkgreen}{rgb}{0,0.4,0}
\definecolor{dk2green}{rgb}{0.4,0,0}


\usepackage[colorlinks,citecolor=darkgreen,linkcolor=darkred,urlcolor=darkblue]{hyperref}
\usepackage{breakurl}
%\usepackage{mathpazo}
\usepackage{enumerate} % for customising enumerated lists

%% For syntax of type theory:
% \usepackage{mathpartir}

%% Graphics and diagrams packages:
\usepackage{tikz}
\usetikzlibrary{arrows}
\usepackage[all]{xy}
\xyoption{2cell}
\xyoption{curve}
\UseTwocells
% \input{diagxy}  % for better inline arrows
% \input{xy-in-tikz} % for arrow tips in TikZ matching xypic’s style
% \tikzset{>=xyto}
\pgfarrowsdeclarealias{c}{c}{right hook}{left hook}
\pgfarrowsdeclarealias{c'}{c'}{left hook}{right hook}

\newbox\pbbox
\setbox\pbbox=\hbox{\xy \POS(65,0)\ar@{-} (0,0) \ar@{-} (65,65)\endxy}
\def\pb{\save[]+<3.5mm,-3.5mm>*{\copy\pbbox} \restore}
% A pullback marker for xymatrix diagrams.
% Usage: add “\pb” in the matrix node of the pullback object, as for instance:
% \xymatrix{ A \ar[r] \ar[d] \pb & B \ar[d] \\ C \ar[r] & D}
%
% To adjust placement in individual diagrams, instead of “\pb” include the
% code following “\def\pb”, and play around with the positioning “<3mm,-3mm>”.

%% Theorem environment declarations (using amsthm):
\theoremstyle{plain}
\newtheorem{theorem}{Theorem}
\newtheorem{axiom}[theorem]{Axiom}
\newtheorem*{theoremstar}{Theorem}
\newtheorem{fact}[theorem]{Fact}
\newtheorem{proposition}[theorem]{Proposition}
\newtheorem{lemma}[theorem]{Lemma}
\newtheorem{corollary}[theorem]{Corollary}

\theoremstyle{definition}
\newtheorem{definition}[theorem]{Definition}
\newtheorem{convention}[theorem]{Convention}
\newtheorem{example}[theorem]{Example}
\newtheorem{examples}[theorem]{Examples}
\newtheorem{notation}[theorem]{Notation}
\newtheorem{remark}[theorem]{Remark}
\newtheorem{idea}[theorem]{Idea}

% Type-theoretic:
\newcommand{\Eq}{\mathsf{Eq}}
\newcommand{\HLevel}{\mathsf{HLevel}}
\newcommand{\hProp}{\mathsf{hProp}}
\newcommand{\Id}{\mathsf{Id}}
\newcommand{\isofhlevel}{\mathsf{isofhlevel}}
\newcommand{\Nat}{\mathsf{Nat}}
\newcommand{\U}{\mathsf{U}}

% Other:
\newcommand{\oftype}{\! : \!}
\newcommand*{\into}{\ensuremath{\lhook\joinrel\relbar\joinrel\rightarrow}}

\makeatletter
\renewcommand{\paragraph}{\@startsection{paragraph}{4}{0mm}{-0.5\baselineskip}{-1ex}{\bf}}
\makeatother

\usepackage[style=authoryear,
  maxnames=2,
 %  maxbibnames=99                  %%%%%%%%     
  uniquename=init
 ]{biblatex}
 \bibliography{references.bib}
\DeclareNameAlias{author}{default}
% \DeclareNameAlias{default}{last-first}
% \DeclareNameAlias{editor}{default}
% \DeclareNameAlias{translator}{default}

\usepackage{hyperref}

\begin{document}
\title{H-Level($n$) is of h-level $n + 1$}

\author{Benedikt Ahrens \and Chris Kapulkin}
%\address[Benedikt Ahrens]{Institute for Advanced Study, Princeton}
%\email{ahrens@ias}

%\author{Chris Kapulkin}
%\address[Chris Kapulkin]{Institute for Advanced Study, Princeton; and University of Pittsburgh}
%\email{krk56@ias.edu}

\date{December 14th, 2012}
\maketitle

In (traditional) mathematics it is well-known that $n$-groupoids and their isomorphisms form an $(n\mbox{+}1)$-groupoid. 
Here, we prove an analogous theorem in Voevodsky's Univalent Foundations, namely, 
that the type of types of h-level $n$ is itself of h-level $n+1$.\footnote{We do not claim much originality: 
a proof of this result is given by \textcite[Section~8.7]{warren-pelayo:univalent-foundations-paper}.}

\paragraph*{The result} We start by fixing a type-theoretic universe $\U$ closed under standard constructors 
(of importance to us are: $\prod$, $\sum$, $\Id$).

Define the type
 \[\HLevel(n) := \sum_{X : \U} \ \isofhlevel(n)(X). \]

Our theorem is:

\begin{theorem}\label{main_thm}
 The type $\HLevel(n)$ is of h-level $n+1$.
\end{theorem}

Before proving the theorem, we give a lemma that can be seen as a version of the Univalence Axiom for predicates on the universe.

\begin{lemma}\label{Id_of_Sigma}
 Let $P \colon \U \to \hProp$ and let $(X, p), (X', p') \oftype \sum\limits_{X : \U} P(X)$. Then:
 \[ \Id_{\sum\limits_{X : \U}P(X)} \big( (X, p), (X', p') \big) \simeq \Eq (X, X'),\]
 where $\Eq (X, X')$ denotes the type of weak equivalences from $X$ to $X'$.
\end{lemma}

\begin{proof}
 We have:
 \begin{equation*}\begin{split}
 \Id_{\sum\limits_{X : \U} P(X)} \big( (X, p), (X', p')\big) & \simeq \sum_{z : \Id_{\U} (X, X')} \Id_{P(X')} (z_!p, p') \\
  & \simeq \Id_{\U} (X, X') \\
  & \simeq \Eq (X, X'),
 \end{split}
 \end{equation*}
 where the first equivalence follows from the fact that a path in a $\sum$-type corresponds to a pair of paths: one in the base type, and one in the fiber; the second is an immediate consequence of $P(X')$ being an h-Prop; and the third is an application of the Univalence Axiom.
\end{proof}

\begin{proof}[Proof of Theorem \ref{main_thm}]
 Let $(X, p), (X', p') \oftype \HLevel(n)$. We need to show that the type $\Id \big( (X, p), (X', p')\big)$ is of h-level $n$. Since for any natural number $n$ and any type $X$ the type: $\isofhlevel(n)(X)$ is an h-Prop, we may apply Lemma \ref{Id_of_Sigma} with $P := \isofhlevel(n)$. Next we observe that it suffices to show that $X \rightarrow X'$ is of h-level $n$ because the projection:
 \[\Eq (X, X') \into (X \rightarrow X').\]
 is an inclusion (i.e. a map of h-level $1$). Since h-levels are preserved under $\prod$ and hence under the arrow type, this reduces to an assumption that $X'$ is of h-level $n$.
\end{proof}

\paragraph*{About the formalization in Univalent Foundations}
We proved this result formally in the proof assistant \textsf{Coq}, based on Voevodsky's
library \textsf{Foundations} \parencite{voevodsky:univalent-foundations-coq}.
In the remainder of this note we present the formal equivalents to the above statements.
We write \lstinline!a == b! for $\Id(a, b)$, and \lstinline!weq X Y! for $\Eq(X, Y)$. The type $\sum_{x : A} B(x)$ is denoted by
\lstinline!total2 (fun x => B x)!.

Voevodsky's library \textsf{Foundations} provides many useful results needed in the proof. However, one important
result was missing: a path in the $\sum$-type is equivalent to a pair of paths consisting of a path in the base and a path in the fiber:
\begin{lstlisting}
Theorem total_paths_equiv (B : UU -> hProp)(x y : total2 (fun x => B x)):
  weq (x == y)
      (total2 (fun p : pr1 x == pr1 y => transportf _ p (pr2 x) == pr2 y )).
\end{lstlisting}
We imported this result from the library \textsf{HoTT} \parencite{hott:repo}; see the file
{\verb+auxiliary_lemmas_HoTT.v+} of our repository \parencite{ahrens-kapulkin:h-level-repo}.

In the file {\verb+HLevel_n_is_of_hlevel_Sn.v+} we prove the above statements.
Lemma \ref{Id_of_Sigma} reads as
\begin{lstlisting}
Lemma Id_p_weq_Id (P : UU -> hProp)(X X' : UU)(pX : P X)(pX' : P X'):
  weq ((tpair _ X pX) == (tpair (fun x => P x) X' pX')) (X == X').
\end{lstlisting}
In the proof of the main theorem \ref{main_thm} we apply the preceding lemma in order to
reduce the goal to the following statement:
\begin{lstlisting}
Lemma isofhlevelpathspace : forall n : nat, forall X Y : UU,
    isofhlevel n X -> isofhlevel n Y -> isofhlevel n (X == Y).
\end{lstlisting}
This statement is proved by distinguishing two cases $n = 0$ and $n > 0$; in the library this is done in
two separate lemmas, called \lstinline!isofhlevel0pathspace! and \lstinline!isofhlevelSnpathspace!,
respectively.

Altogether, the two files comprise about 80 lines of specification and 150 lines of proof, 40\% of which
are used for the translation of \lstinline!total_paths_equiv! from the \textsf{HoTT} library to
\textsf{Foundations}.
Our \textsf{Coq} files (along with the necessary files from Voevodsky's \textsf{Foundations} repository) can be found at
\begin{center}
 \url{https://github.com/benediktahrens/HLevel_n_is_of_hlevel_Sn}
\end{center}

% \bibliographystyle{alpha}
\printbibliography

\end{document} 