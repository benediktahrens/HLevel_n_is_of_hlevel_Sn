\documentclass[11pt]{amsart}

%% Input handling:
\usepackage[utf8]{inputenc}  % to allow unicode in source

%% AMS and other general math packages:
\usepackage{amsthm}
\usepackage{amsfonts}
\usepackage{amsmath}
\usepackage{amssymb}
% \usepackage{mathtools} % for \mathclap, used to center extra-wide diagrams.

%% General style packages:
\usepackage{xcolor}
\definecolor{darkgreen}{rgb}{0,0.45,0}
\definecolor{darkred}{rgb}{0.75,0,0}
\definecolor{darkblue}{rgb}{0,0,0.6}
\usepackage[colorlinks,citecolor=darkgreen,linkcolor=darkred,urlcolor=darkblue]{hyperref}
\usepackage{breakurl}
%\usepackage{mathpazo}
\usepackage{enumerate} % for customising enumerated lists

%% For syntax of type theory:
% \usepackage{mathpartir}

%% Graphics and diagrams packages:
\usepackage{tikz}
\usetikzlibrary{arrows}
\usepackage[all]{xy}
\xyoption{2cell}
\xyoption{curve}
\UseTwocells
% \input{diagxy}  % for better inline arrows
% \input{xy-in-tikz} % for arrow tips in TikZ matching xypic�s style
% \tikzset{>=xyto}
\pgfarrowsdeclarealias{c}{c}{right hook}{left hook}
\pgfarrowsdeclarealias{c'}{c'}{left hook}{right hook}

\newbox\pbbox
\setbox\pbbox=\hbox{\xy \POS(65,0)\ar@{-} (0,0) \ar@{-} (65,65)\endxy}
\def\pb{\save[]+<3.5mm,-3.5mm>*{\copy\pbbox} \restore}
% A pullback marker for xymatrix diagrams.
% Usage: add �\pb� in the matrix node of the pullback object, as for instance:
% \xymatrix{ A \ar[r] \ar[d] \pb & B \ar[d] \\ C \ar[r] & D}
%
% To adjust placement in individual diagrams, instead of �\pb� include the
% code following �\def\pb�, and play around with the positioning �<3mm,-3mm>�.

%% Theorem environment declarations (using amsthm):
\theoremstyle{plain}
\newtheorem{theorem}{Theorem}
\newtheorem{axiom}[theorem]{Axiom}
\newtheorem*{theoremstar}{Theorem}
\newtheorem{fact}[theorem]{Fact}
\newtheorem{proposition}[theorem]{Proposition}
\newtheorem{lemma}[theorem]{Lemma}
\newtheorem{corollary}[theorem]{Corollary}

\theoremstyle{definition}
\newtheorem{definition}[theorem]{Definition}
\newtheorem{convention}[theorem]{Convention}
\newtheorem{example}[theorem]{Example}
\newtheorem{examples}[theorem]{Examples}
\newtheorem{notation}[theorem]{Notation}
\newtheorem{remark}[theorem]{Remark}
\newtheorem{idea}[theorem]{Idea}

% Type-theoretic:
\newcommand{\Eq}{\mathsf{Eq}}
\newcommand{\HLevel}{\mathsf{HLevel}}
\newcommand{\hProp}{\mathsf{hProp}}
\newcommand{\Id}{\mathsf{Id}}
\newcommand{\isofhlevel}{\mathsf{isofhlevel}}
\newcommand{\Nat}{\mathsf{Nat}}
\newcommand{\sfPi}{\mathsf{\Pi}}
\newcommand{\sfSigma}{\mathsf{\Sigma}}
\newcommand{\U}{\mathsf{U}}

% Other:
\newcommand{\oftype}{\! : \!}
\newcommand*{\into}{\ensuremath{\lhook\joinrel\relbar\joinrel\rightarrow}}

\makeatletter
\renewcommand{\paragraph}{\@startsection{paragraph}{4}{0mm}{-0.5\baselineskip}{-1ex}{\bf}}
\makeatother


\begin{document}
\title{H-Level $n$ is of h-level $n\mbox{+}1$}

\author{Benedikt Ahrens}
% \address[Benedikt Ahrens]{Institute for Advanced Study, Princeton}
% \email{ahrens@ias.edu}

\author{Chris Kapulkin}
% \address[Chris Kapulkin]{Institute for Advanced Study, Princeton; and University of Pittsburgh}
% \email{krk56@ias.edu}

\date{December 3rd, 2012}
\maketitle

In this paper we prove that the type of types of h-level $n$ is itself of h-level $n+1$. That mirrors a familiar situation from higher category theory, where one has that the category of $n$-categories is itself an ($n\mbox{+}1$)-category.

We fix a type-theoretic universe $\U$ closed under standard constructors. Define the type:
 \[\HLevel(n) := \sfSigma_{X : \U} \ \isofhlevel \ n \ X \]

\begin{theorem}\label{main_thm}
 The type $\HLevel(n)$ is of h-level $n+1$.
\end{theorem}

\begin{lemma}\label{Id_of_Sigma}
 Let $P \colon \U \to \hProp$ and let $(X, p), (X', p') \oftype \sfSigma_{X : \U} P(X)$. Then:
 \[ \Id_{\sfSigma_{X : \U} P(X)} ((X, p), (X', p')) \simeq \Id_{\U} (X, X').\]
\end{lemma}

\begin{proof}
 We have:
 \begin{equation*}\begin{split}
  \Id_{\sfSigma_{X : \U} P(X)} ((X, p), (X', p')) & \simeq \sfSigma_{z : \Id_{\U}(X, X')} \Id_{P(X')}(z_!p, p') \\
  & \simeq \Id_{\U} (X, X')
 \end{split}
 \end{equation*}
\end{proof}

\begin{proof}[Proof of Theorem \ref{main_thm}]
 Let $(X, p), (X', p') \oftype \HLevel(n)$. We need to show that the type:
 \[ \Id_{\sfSigma_{X : \U} \isofhlevel \ n \ X} ((X, p), (X', p')) \]
 is of h-level $n$. Since for any natural number $n$ and any type $X$ the type:
 \[\isofhlevel \ n \ X \]
 is an hProp, we may apply Lemma \ref{Id_of_Sigma} with $P := \isofhlevel \ n$. In order to show that $\Id_{\U} (X, X')$ is of h-level $n$ we proceed as follows. By the Univalence Axiom, it suffices to show that that $\Eq(X, X')$ is of h-level $n$. We have an inclusion:
 \[\Eq (X, X') \into X \rightarrow X'\]
 i.e. a map of h-level $1$. Thus it suffices to show that $X \rightarrow X'$ is of h-level $n$. Since h-levels are preserved under $\sfPi$ and hence under the arrow type, this reduces to an assumption that $X'$ is of h-level $n$.
\end{proof}

\noindent
Find the accompanying \textsf{Coq} code on
\begin{center}
\url{https://github.com/benediktahrens/HLevel_n_is_of_hlevel_Sn} .
\end{center}

\end{document} 